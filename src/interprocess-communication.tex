\section{Inter-process Communication}
Two processes $P_1$ and $P_2$ need to communicate with each other. There are two ways to do this:
\begin{enumerate*}
    \keyitem{shared memory}{$P_1$ creates a shared memory region which $P_2$ can attach to.
    $P_1$ and $P_2$ can then read and write to the shared memory region.}
    \keyitem{message passing}{$P_1$ and $P_2$ exchange messages, where sending and receiving involve the OS via syscalls.}
\end{enumerate*}

The OS overhead for message passing is much higher than for shared memory, where the OS is only involved during the setup and teardown.

Both options require some form of synchronization, to avoid \textbf{data races},
where the processes read and write to the same memory location at the same time.

Race conditions may result in incorrect behavior.

\begin{defn}{direct \& indirect communication}
    In a shared memory scheme, communication is always \textit{implicit} --- a bystander cannot easily detect that communication is taking place.

    However, in message passing, communication is \textit{explicit} --- the sender must explicitly name the recipient, or specify a common message storage, usually a shared \textbf{mailbox} or \textbf{port}.

    This means message passing may require the use of finite \textit{kernel memory} to buffer messages.
\end{defn}

\begin{defn}{synchronous \& asynchronous message passing}
    The \code{receive()} operation is always assumed to be \textit{blocking} --- always waiting until a message arrives.
    
    Asynchronous message passing is \textit{non-blocking}, requiring the kernel to buffer messages. In the case of buffer overflow, \code{send()} may:
    \begin{itemize}
        \item block until the buffer has space, or,
        \item return an error.
    \end{itemize}

    Synchronous message passing is \textit{blocking}, requiring the sender to wait until the receiver has received the message. No buffering is required.
\end{defn}

\subsection{Example: Unix IPC}

\begin{defn*}{shared memory}
    \begin{enumerate*}
        \item The master program creates the shared memory space via \code{shmget()} and attaches it via \code{shmat()}.
        \item The worker process only needs to attach to the shared memory space via \code{shmat()}.
        \item Both processes use some space in the shared memory as control values for synchronization.
        \item After work is complete, the worker process detaches from the shared memory via \code{shmdt()}.
        \item The master process detaches from the shared memory via \code{shmdt()} and destroys the shared memory via \code{shmctl()}.
    \end{enumerate*}
\end{defn*}

\begin{defn*}{message passing --- pipes}
    Every process has \code{stdin}, \code{stdout}, and \code{stderr} pipes,
    which can redirected with \code{<}, \code{>}, and \code{2>} respectively.
    
    Pipes in a \textbf{producer-consumer} relationship are connected with \code{|} at the command line.

    Pipes are created with the \code{pipe(fd[2])} syscall in C, which returns two file descriptors, one for reading and one for writing.

    The producer must \code{close()} the read end of the pipe before writing to it, and then \code{close()} the write end of the pipe after writing to it.

    The consumer must \code{close()} the write end of the pipe before reading from it, and then \code{close()} the read end of the pipe after reading from it.
\end{defn*}

\begin{defn}{signals}
    Signals are asynchronous notifications sent to a process by the OS or another process.

    Processes can define their own signal handlers to handle signals,
    such as \code{SIGSEGV}, using the \code{signal(SIGNAL, *fp)} syscall in C.

    The \code{SIGKILL} and \code{SIGSTOP} signals are sent by the kernel to a process to terminate it, and cannot be caught, blocked, or ignored.
\end{defn}