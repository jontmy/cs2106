\part{File Systems}
\textbf{Files} can be viewed as a collection of logical blocks.
When the file size is not a multiple of the block size, the last block may have \textbf{internal fragmentation}.

\textbf{File systems} keep track of these blocks, allow efficient access, and ensure effective utilization of disk space.


\section{File System Implementation}
Each partition on a disk contains 5 sectors:
\begin{enumerate}
    \keyitem*{OS boot block}{}
    \keyitem*{partition details}{stores free space information}
    \keyitem*{directory structure}{stores directory information}
    \keyitem*{file info}{stores file allocation information}
    \item{\textbf{file data}}
\end{enumerate}

\subsection{Allocation}
In \textbf{contiguous allocation}, blocks are stored in a contiguous 1D array with metadata on the starting indices and length (in blocks) of each file.

\textit{Advantage}: efficient sequential and random access as offsets into a file can be used.

\textit{Disadvantage}: starting location of each file is assigned randomly, leading to external fragementation (empty blocks between files).

In \textbf{linked list allocation}, each block in a file points to the next block in the file, and the addresses of the starting and ending blocks are recorded.

\textit{Advantage}: no external fragmentation.

\textit{Disadvantage}: pointer overhead per block, possible pointer corruption, sequential and random access is inefficient as the links must be traversed to reach the desired block.

\begin{defn}{file allocation table (FAT)}
    In a file allocation table, each block has an entry and is paired with the index of the next block in the file.

    The FAT can be stored in RAM which makes random access efficient, but the FAT must be updated when a file is modified.

    Furthermore, the FAT must record all disk blocks in a prtition, which on large disks can be memory space-inefficient.
\end{defn}

In \textbf{indexed allocation}, the index of every block in a file is also stored in a block, such that only the index block needs to be read to access the file.

\textit{Advantage}: less memory overhead, only index block is kept in memory, no external fragmentation.

\textit{Disadvantage}: index block overhead, maximum file size is limited by the number of blocks in the index block.

Variations of the indexed allocation strategy include \textbf{linked indexes} and \textbf{multi-level indexing}.


\subsection{Free Space Management}
Using a \textbf{bitmap}, each block is marked as either free or allocated with \code{0} or \code{1} (implementation-dependent).

\textit{Advantage}: bitwise manipulation is efficient.

\textit{Disadvantage}: memory overhead as the bitmap must be stored in memory for efficiency.

Using a \textbf{linked list}, free blocks are linked together and this information is stored in a block. Free space can be allocated in reverse order of the linked list.

\textit{Advantage}: only the first pointer needs to be stored in memory, and free blocks are easy to locate.

\textit{Disadvantage}: high overhead if not storing the free block list in a free block.


\subsection{Directory Structure}
Directories are used to group files together, and are actually files themselves.
They contain information of the files and sub-directories in the directory.

In a \textbf{linear list} implementation, each file and its starting index and length is stored in a list.

\textit{Disadvantage}: linear search to find a file is inefficient, caching can offset this

In a \textbf{hash table} implementation, file names are hashed to a bucket, and their starting index and length are stored in a list in that bucket.

\textit{Advantage}: fast lookup.

\textit{Disadvantage}: hash collisions, chained collision resolution will lead to linear search over the length of the chain.

Another approach to storing file information is to use pointers to the metadata instead of storing the metadata directly.


\subsection{Microsoft FAT File System}
Each partition on a disk contains 5 sectors:
\begin{enumerate}
    \item{\textbf{boot block}}
    \keyitem*{FAT}{stores partition details}
    \keyitem*{duplicate FAT}{backup in case of FAT corruption}
    \keyitem*{root directory}{stores all file and directory information (FAT16 only)}
    \keyitem*{data blocks}{stores directory structure, file info and file data}
\end{enumerate}

Every entry in the FAT contains one of the following:
\begin{enumerate}
    \item block number of the next block
    \item \code{EOF} code (e.g. \code{NULL} pointer)
    \item \code{FREE} code (unused block)
    \item \code{BAD} code (unusuable block, e.g. disk error)
\end{enumerate}

For FAT16, there are up to $2^{16}$ blocks and each of the $2^{16}$ FAT entries is $16$ bits.

For FAT32, there are up to $2^{32}$ blocks and each of the $2^{32}$ FAT entries is $32$ bits.

\begin{defn}{directory entries}
    The root directory has its own sector in the partition, and other directories are stored in the data blocks of the partition.

    Each file or subdirectory is stored as a \textbf{directory entry} and consists of:
    \begin{enumerate}
        \item file name, 8 bytes
        \item file extension, 3 bytes
        \item file attributes, 1 byte
        \item reserved bytes, 10 bytes
        \item creation date and time, 4 bytes
        \item first disk block number, 2 bytes
        \item file size in bytes, 4 bytes
    \end{enumerate}

    \textbf{File attributes} are used to indicate whether a file is a file, directory, or special (e.g. read only, hidden, etc.).

    The FAT version can be determined by the size of the first disk block number --- $2$ bytes for FAT16 and $4$ bytes for FAT32, hence the reserved space.
\end{defn}

Since the file name is limited 8 characters, in FAT32, longer file names are stored in separate directory entries. A tilde is used to denote that the file name continues into the next directory entry.

\begin{defn}{reading a file in FAT}
    Given the root directory, data blocks and FAT sectors, find the sub-directory within the root directory, and then find the directory entry for the file within the sub-directory.

    Alternate between the data block and FAT table starting from the first disk block number until the \code{EOF} code is reached.
\end{defn}


\subsection{Extended-2 (Ext2) File System}
Each partition is divided into the boot sector followed by multiple \textbf{block groups}.

Each block group contains:
\begin{enumerate}
    \keyitem{superblock}{metadata for the file system, e.g. how many block groups, how many I-nodes per group, etc.}
    \keyitem{group descriptors}{metadata on where to find the following items, and duplicates all other group descriptors for redundancy}
    \keyitem{block bitmap}{one bit for every data block, \code{1} for allocated and \code{0} for free blocks}
    \keyitem{I-node bitmap}{one bit for every I-node, \code{1} for allocated and \code{0} for free I-nodes}
    \keyitem{I-node table}{one 128-bytes entry for every file and directory, contains file information}
    \keyitem{data blocks}{contain file data and directory structure}
\end{enumerate}

\begin{defn}{I-nodes (index-nodes)}
    Each I-node has the following structure:
    \begin{enumerate}
        \item mode (file type, permission bits, etc.), 2 bytes
        \item owner information, 4 bytes
        \item file size in bytes, 4 or 8 bytes
        \item timestamps, 3 x 4 bytes
        \item \textbf{data block pointers}, 15 x 4 bytes
        \item \textbf{reference count}, 2 bytes
    \end{enumerate}

    I-nodes use the combined allocation strategy with linked indices and multi-level indexing.

    The first 12 pointers of the data block pointers point to \textbf{direct blocks} of 1 KiB each storing actual data.

    The 13th pointer points to a \textbf{single indirect block}, which uses single-level indexing to point to the next 256 1KiB blocks.

    (256 entries as the indirect block is also 1 KiB, and each address is 4 bytes.)

    The 14th pointer points to a \textbf{double indirect block}, which uses double-level indexing for a maximum of 64 MiB.
    
    The 15th pointer points to a \textbf{triple indirect block}, which uses triple-level indexing for a maximum of 16 GiB.

    The \textbf{reference count} is used to determine when to free the I-node and its data blocks.
\end{defn}

With multiple levels of indexing, both small and huge files can be handled and accessed efficiently.

\begin{defn}{directory structure}
    Directories are stored as I-nodes, and the data blocks of the I-node point to the files and sub-directories, forming \textbf{directory entries}.

    Directory entries are stored in a linked list and each has associated with it:
    \begin{enumerate}
        \item an I-node number for the file or sub-directory,
        \item an offset to skip to the next directory entry,
        \item a flag for the file type, 'F' for file and 'D' for directory,
        \item the size of the file in bytes,
        \item and a variable-length file name.
    \end{enumerate}

    Each directory entry is variable length due to the variable-length file names.

    The final directory entry points to an I-node number of \code{0}.
\end{defn}

Creating hard links involves new directory entries pointing to the same I-node, increasing the reference count of that I-node.

Creating symlinks incurs the overhead of new I-nodes and data blocks, to store the path to the file (incurring re-traversal overhead), but does not increment the reference count.

\begin{defn}{reading a file in Ext2}
    Given the I-node number for the root directory and the I-node table:

    \begin{enumerate}
        \item Find the I-node for the root directory.
        \item Find the disk block number for the directory entries for the root directory from the I-node.
        \item Access the disk block and find the I-node number of the sub-directory/file.
        \item Repeat steps 2 and 3 until the I-node number of the file is found.
        \item Access the disk blocks in the I-node to read the file data.
    \end{enumerate}
\end{defn}