\section{Process Abstraction}

A \textbf{process} abstracts the information required to describe and manage the execution of a program.

\subsection{Function Invocation}
Function invocation presents control flow and data storage challenges:
\begin{itemize}
    \item jumping to function body, and resuming after the function returns
    \item passing  parameters, and capturing a return value
    \item allocating local variables
\end{itemize}

\begin{defn}{stack memory}
    Stack memory is a region of memory separate from instruction and data memory that is used to store the following information for function invocations in \textbf{stack frames}:
    \begin{enumerate}
        \item return address (\code{PC}) of the caller
        \item arguments for the parameters of the function
        \item local variables used by the function
        \item saved \textbf{stack pointer (\code{SP})}
    \end{enumerate}

    Optionally, the stack frame also includes:
    \begin{enumerate}
        \item[5.] \textbf{frame pointer \code{(FP)}}
        \item[6.] saved registers
    \end{enumerate}
\end{defn}

Stack frames are \textbf{pushed} onto the stack on invocation and \textbf{popped} on return.

The \textbf{stack pointer \code{(SP)}} is a special register which contains the address of the first free location in the stack.

Because \code{SP} varies depending on the number of local variables, some processors provide a \textbf{frame pointer \code{(FP)}} which points to a fixed location in a stack frame.

When all general purpose registers are exhausted (\textbf{register spilling}), their values can be written to memory and restored at the end of the function call.

\begin{defn}{function invocation --- setup}
    If a function \code{f()} invokes \code{g()}, then \code{f()} is the \textbf{caller} and \code{g()} is the \textbf{callee}.

    \begin{itemize}
        \keyitem*{caller}{
            \begin{enumerate}
                \item passes arguments by setting registers directly and/or pushes arguments onto the stack
                \item saves the return address on the stack
            \end{enumerate}
        }
        \keyitem*{callee}{
            \begin{enumerate}
                \item saves the old stack pointer
                \item allocates space for local variables
                \item sets \code{SP} to the top of the stack
            \end{enumerate}
        }
    \end{itemize}
\end{defn}

\begin{defn*}{function invocation --- teardown}
    \begin{itemize}
        \keyitem*{callee}{
            \begin{enumerate}
                \item pushes return value onto the stack (if any)
                \item restores \code{SP} to the saved stack pointer
                \item sets \code{PC} to the return address
            \end{enumerate}
        }
        \keyitem*{caller}{
            \begin{enumerate}
                \item uses the return result (if any)
                \item resumes execution
            \end{enumerate}
        }
    \end{itemize}
\end{defn*}

Actual implementation of the function call convention depends on hardware, programming language, and compiler --- it is not universal.

\subsection{Dynamically Allocated Memory}
Some data may have a size only known at runtime, which precludes allocation in the data memory.

Some data may have an unknown lifetime (i.e. no deallocation time guarantees) which precludes allocation in the stack memory.

Hence, such dynamic data is allocated (e.g. by \code{malloc}) in the \textbf{heap memory}.

\subsection{Process Model}
\textbf{Process IDs} (PIDs) uniquely identify processes, each of which may exist in 5 different possible states:

\begin{enumerate}
    \keyitem*{new}{created, or under initialization}
    \keyitem*{ready}{admitted, but waiting to run}
    \keyitem*{running}{being executed by the CPU}
    \keyitem*{blocked}{waiting until an event occurs}
    \keyitem*{terminated}{finished execution, possibly requiring OS cleanup}
\end{enumerate}

\begin{defn}{process control blocks (PCBs)}
    Every process' registers, memory, PID, and state is stored in a \textbf{process control block}.
    
    The kernel manages all PCBs in a \textbf{process table}.
\end{defn}

\begin{defn}{context switching}
    Programs may be interrupted at any time, e.g. by the \textbf{scheduler},
    and must operate independently of interrupts.
    
    A \textbf{context switch} saves the current execution context process in a PCB and loads the context of the next process to be executed.
\end{defn}

