\part{Memory Management}

\section{Contiguous Memory Management}
Each process occupies a contiguous memory region and the physical memory must be large enough to contain one or more processes with complete memory space.

A \textbf{memory partition} is a contiguous memory region allocated to a process.

\subsection{Fixed Partitioning}
Every partition is of the same size, occupied by at most one process.

\textbf{Internal fragmentation} occurs when a process does not occupy the entire partition.

\begin{itemize}
    \item[+] Easy to manage.
    \item[+] Fast allocation as every free partition is the same size, no choosing required.
    \item[-] Partition size must be larger than or equal to the size of the largest process, leading to internal fragmentation.
\end{itemize}

\subsection{Dynamic Partitioning}
Free memory spaces are known as \textbf{holes}, which are filled by processes as they arrive.

These holes are tracked in a \textbf{free list}, where they can be \textbf{merged} or \textbf{compacted} (moving processes to consolidate holes).

\textbf{External fragmentation} is a result of the holes which are too small to be used by any process.

\begin{itemize}
    \item[+] No internal fragmentation.
    \item[-] Slow allocation as the entire memory must be searched for a suitable hole.
    \item[-] Overhead of maintaining information on the holes.
    \item[-] External fragmentation.
\end{itemize}

\textbf{Allocation algorithms} are used to decide which hole to fill to allocate a partition.

\begin{itemize}
    \keyitem*{first-fit}{the first hole that is large enough}
    \keyitem*{best-fit}{the smallest hole that is large enough}
    \keyitem*{worst-fit}{the largest hole}
    \keyitem*{quick-fit}{maintain multiple free lists of holes with exponentially increasing sizes}
    \keyitem*{buddy system}{contiguous block of memory is recursively sub-divided into two halves into the smallest possible block size able to fit the process}
\end{itemize}

After a partition is allocated, the leftover space becomes a new hole.

\section{Disjoint Memory Management}
We no longer assume that processes occupy contiguous memory regions.


\subsection{Paging}
The \textbf{logical memory} of a process is divided into \textbf{logical pages}, which are mapped onto \textbf{physical memory} split into regions of fixed size known as \textbf{physical frames}.

Logical pages and physical frames are generally fixed to the same size, and typically a power of 2.

This translation of logical pages to physical frames is stored in a \textbf{page table} \textit{for each process}. Page tables are stored in physical memory.

\begin{itemize}
    \item[+] No external fragmentation as every frame can be used.
    \item[+] Insignificant internal fragmentation as at most one page is not fully used. 
    \item[-] Memory reference requires two memory accesses: one to the page table and one to the physical frame.
    \item[-] Page tables can get very large.
\end{itemize}

\begin{defn}{page table translation}
    Each logical memory address is split into two parts: the \textbf{page number} (MSBs) and the \textbf{offset} (LSBs).

    Each \textbf{page table entry} (PTE) stores the physical frame number of the corresponding logical page number.
    
    Given a page and frame size of $2^n$ bytes and an $m$ bit logical address, the page number is the $m - n$ MSBs of the logical address, and the offset is the $n$ LSBs.

    The $n$ LSBs of the physical address are the same as the offset, and the $m - n$ MSBs are the physical frame number found using the corresponding PTE.
\end{defn}

If a page table translation results in incorrect access to a physical address that is still within a frame that is mapped to the process, this error cannot be caught.

Page table entries can be augmented with \textbf{access-right bits} and \textbf{valid bits}.

Access-right bits (for \textit{writable}, \textit{readable}, and \textit{executable}) allow hardware to check that a process is allowed to access a page.

Valid bits are set by the OS during allocation.
Every memory access is checked in hardware to catch out-of-bounds accesses.

Pages can also be shared with a \textbf{copy-on-write} mechanism, typically implemented using \textbf{reference counting}.


\subsection{Translation Look-aside Buffers (TLBs)}
A TLB is a hardware cache storing a small number of page table entries which can be read in under a clock cycle.

On a \textbf{TLB hit}, the frame number can be retrieved directly from the TLB and used to access the physical frame without needing to access the in-memory page table (memory access is slow).

On a \textbf{TLB miss}, the page table must be accessed to retrieve the frame number, after which it is stored in the TLB.

As such, memory latency is lower with a TLB by bypassing the page table.

However, the TLB must be filled to be effective, and hence the initial load or context switch to a process will result in many TLB misses.

TLBs are flushed on a context switch to prevent incorrect translations and for memory security, which means OSes have to be aware of the TLB.


\subsection{Segmentation}
The logical memory of a process is contiguous, which is not always desirable.

With segmentation, the logical memory is divided into \textbf{segments},
giving each segment its own independent logical address space, permissions, lifetime, and size.

Each segment has a \textbf{name} and \textbf{limit} and is mapped onto physical memory with a \textbf{base address} and \textbf{limit/size}.

These base addresses and limits are stored in a \textbf{segment table} for each segment
of a process, and a \textbf{segment ID} is used to identify each segment name.

\begin{itemize}
    \item[+] Segments can grow and shrink independently.
    \item[+] Segments can be shared independently between processes.
    \item[-] External fragmentation as each segment requires a variable-sized and contiguous region of physical memory.
\end{itemize}

\begin{defn}{logical address translation}
    A logical address takes the form \code{<segment ID, offset>}.

    The segment ID is used to find the corresponding segment table entry, which contains the base address and limit of the segment.

    The offset is added to the base address to get the physical address, after which only valid accesses are enforced requiring $offset < limit$.
\end{defn}

Segmentation can be used with paging.

Each segment can span multiple pages, and as such maintains its own page table.

In addition, some machines can provide dedicated \textbf{base registers} to store the base addresses and limits of segments, eliminating the need for a TLB or any other page translation caching mechanism.