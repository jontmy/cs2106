\section{Process Scheduling}
To execute multiple processes concurrently, instructions from processes on the same CPU core are interleaved
in a process known as \textbf{timeslicing}.

The \textbf{scheduler} is the part of the OS which makes selects processes to run using \textbf{scheduling algorithms} to ensure \textit{fairness} (no starvation) and \textit{full utilization} of the CPU.

There are 2 kinds of scheduling policies:
\begin{enumerate}
    \keyitem{preemptive}{processes are given a fixed time quota, and are interrupted if they do not block or terminate}
    \keyitem{non-preemptive}{processes remain in the running state until they block or terminate \textit{voluntarily}}
\end{enumerate}

Processes go through phases of \textbf{CPU activity} and \textbf{IO activity} and their processing environment falls into 3 categories:
\begin{enumerate}
    \keyitem{batch processing}{no user interaction, need not be responsive}
    \keyitem{interactive}{must have a low and consistent response time}
    \keyitem{real-time processing}{deadlines must be met, and usually periodic}
\end{enumerate}

\subsection{Batch Processing Scheduling}
Batch processing are evaluated on 4 metrics:
\begin{itemize}
    \keyitem*{turnaround time}{finish time - arrival time}
    \keyitem*{throughput}{number of tasks finished per unit time}
    \keyitem*{makespan}{total time taken to complete all tasks}
    \keyitem*{CPU utilization}{percentage of time the CPU is working}
\end{itemize}

\begin{defn}{first come first serve (FCFS)}
    Tasks are stoped in a FIFO queue, enqueued when they arrive and dequeued to run until they block or terminate.

    Guarantees \textbf{no starvation} as the number of tasks is strictly decreasing.

    Suffers from the \textbf{convoy effect} --- a CPU-bound process blocks queuing IO-bound processes, which leads to I/O idling, after unblocking, the IO-bound processes only require low CPU utilization.
\end{defn}

A simple reordering of processes in the queue can reduce average waiting time.

\begin{defn}{shortest job first (SJF)}
    Tasks are ordered in a priority queue with the task with the least CPU time estimated given the highest priority.

    Estimation is typically done using an exponential moving average:

    $\text{Predicted}_{n+1} = \alpha \; \text{Actual}_n + (1 - \alpha) \; \text{Predicted}_n$, \\
    where $\alpha$ is the weight given based on past history.

    Allows \textbf{preempting} a running task if a task with a shorter job arrives.

    Guarantees smallest average waiting time.

    Suffers from possible \textbf{starvation}.
\end{defn}

\begin{defn}{shortest remaining time (SRT)}
    Similar to SJF, except priority is given to the task with the least remaining CPU time.

    Allows \textbf{preempting} a running task if a task with a shorter remaining time arrives.

    Suffers from possible \textbf{starvation}.
\end{defn}

\subsection{Interactive Scheduling}

Interactive environments are evaluated by their \textbf{response time} and \textbf{predictability}, which is the variation in response time.

A timer interrupt (which the OS ensures cannot be intercepted) is raised at fixed intervals and its handler invokes the scheduler.

This divides CPU time into \textbf{time quantums}, which are multiples of the timer interrupt interval.

\begin{defn}{round robin (RR)}
    RR is the preemptive version of FCFS, where tasks in the FIFO queue are given a fixed time quantum to run,
    after which they are preempted and placed at the end of the queue.

    Smaller time quantums trade CPU utilization for better response time.
\end{defn}

\begin{defn}{lottery scheduling}
    A weighted random process is chosen to run during each time quantum.
    
    The weight determines the probability of being chosen and ultimately the CPU time for that process.

    Allows a process to distribute its CPU time across multiple child processes.
\end{defn}

\begin{defn}{priority scheduling}
    Processes are assigned priority values, and the scheduler selects the process with the highest priority to run, either preemptive or non-preemptive.

    Suffers from possible \textbf{starvation} and \textbf{priority inversion}, where a high priority process is blocked by a low priority process, e.g. due to a lock on a shared resource.
    
    Decreasing priority over time can mitigate starvation.

    \textbf{Priority inheritance} can be used to mitigate priority inversion, where a lower priority process that is locking a shared resource is temporarily given the priority of the higher priority process that is blocked on it until it unlocks it.
\end{defn}

\begin{defn}{multi-level feedback queue (MLFQ)}
    MLFQ is a priority scheduling algorithm with multiple queues, where processes are moved between queues based on their CPU usage.

    New jobs get the highest priority, and are moved to lower priority queues if they fully utilize their time quantum.

    If a job gives up or blocks before fully utilizing its time quantum, it retains its priority.

    Processes with equal priority are scheduled using RR.
\end{defn}